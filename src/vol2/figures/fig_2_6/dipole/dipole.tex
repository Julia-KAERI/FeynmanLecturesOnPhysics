% Author: Izaak Neutelings (February 2020)
\documentclass[border=3pt,tikz]{standalone}
\usepackage{physics}
\usepackage{xcolor}
\usetikzlibrary{decorations.markings}
\tikzset{>=latex} % for LaTeX arrow head

\colorlet{Ecol}{orange!90!black}
\colorlet{Bcol}{violet!90}
\colorlet{Icol}{blue!70!black}
\colorlet{gausscol}{green!40!black}
\colorlet{gausscol2}{green!45!blue}
\tikzstyle{current}=[->,Icol,thick]
\colorlet{pluscol}{red!60!black}
\colorlet{minuscol}{blue!60!black}
\tikzstyle{anode}=[top color=red!20,bottom color=red!50,shading angle=20]
\tikzstyle{cathode}=[top color=blue!20,bottom color=blue!40,shading angle=20]
\tikzstyle{gauss surf}=[gausscol,top color=green!2,bottom color=green!80!black!70,shading angle=5,fill opacity=0.4]
\tikzstyle{metal}=[top color=black!15,bottom color=black!25,middle color=black!20,shading angle=10]
\tikzstyle{mydashes}=[dash pattern=on 1 off 1]
\tikzset{
  EFieldLine/.style={thick,Ecol,line cap=round,decoration={markings,
                     mark=at position #1 with {\arrow{latex}}},
                     postaction={decorate}},
  BFieldLine/.style={thick,Bcol,postaction={decorate},decoration={markings,
                     mark=at position #1 with {\arrow{latex}},
                     mark=at position #1+0.5 with {\arrow{latex}}}},
  EFieldLine/.default=0.5,
  BFieldLine/.default=0.4}
\usetikzlibrary{3d}

\begin{document}


% CAPACITOR 3D - displacement current derivation
\begin{tikzpicture}[scale=2]
    \def\R{3}
    \coordinate (P) at (50:\R);
    \draw [thick, pluscol, ->] (0,0) --node [right] {$\boldsymbol{r}$} (P);
    \draw[thick, -latex] (-1, 0) -- (3, 0) node [below] {$y$};
    \draw[thick, -latex] (0, -1) -- (0, 3) node [left] {$z$};
    \draw[thick, -latex]  (2.7, 1.8) -- (-0.6, -0.4) node [left] {$x$};
    \filldraw[pluscol] (0, 0.4) circle (1pt) node [right] {$+q$};
    \filldraw[minuscol] (0, -0.4) circle (1pt) node [right] {$-q$};
    \draw[gausscol, <->, thick] (-0.2, 0.4) -- node[left, fill=white] {$d$} (-0.2, -0.4);
    \draw[<->] (0, 1) arc (90:50:1) node[midway, above] {$\theta$}; 
\end{tikzpicture}
  
\end{document}